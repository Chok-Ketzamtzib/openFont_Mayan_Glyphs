%%%==================================================================
%%%
%%%   mayaps.tex    version 1.1b   March 4, 2017
%%%                 works with  mayaps.pro  version 1.1b
%%%
%%%   For typing ancient Maya glyphs with TeX/LaTeX  AND(!!!)  dvips. 
%%%
%%%   author:  Stepan Orevkov  <orevkov(a)math.ups-tlse.fr>
%%%
%%%   Acknowlegments:
%%%            1. Ancient Maya typesetting principles are due to
%%%               Bruno Delprat
%%%            2. The TeX code in this file is inspired mostly
%%%               by `epsf.tex' (Tom Rokicki)
%%%            3. The idea to use PS for drawing composed glyphs
%%%               belongs to Ilya Zakharevich
%%%
%%%   Changes:
%%%
%%%   version 0.13  1. checking balanced parentheses
%%%                 2. selecting only those glyphs which are used
%%%                    in the document
%%%
%%%   version 0.14  1. multifont support (macro \mayaFont\cs=fontname)
%%%                 2. add glyphs from eps files (macro \mayaAddGlyph)
%%%
%%%   version 0.15  1. add braces around \def's in \mayaAddGlyph because
%%%                    otherwise \c{c} (c-cedille) is not printed
%%%                 2. add modifiers '*' and '?' which are equivalent to
%%%                    'r' and 'R' respectively but do not require []
%%%                 3. see files makefont and empty.mpf
%%%
%%%   version 0.16  1. the affix modification vector [(L)(U)(R)(D)] is
%%%                    extended up to [(L)(U)(R)(D)(S)] where (S) is the
%%%                    string of modifiers applied for a single affix
%%%                 2. make \mayaAddGlyph to replace glyphs (is was
%%%                    declared in previous versions but it did not work)
%%%
%%%   version 0.17  1. add modifier '&' which is equivalent to 'c' but
%%%                    does not requires []
%%%                 2. when an affix is converted to a central element,
%%%                    then its default orientation is (S)
%%%                 3. new algorithm for affix orientation choice
%%%                    (see epmty.mpf)
%%%
%%%   version 0.18  1. ligatures
%%%
%%%   version 0.19  1. correct \maya@L. Now it checks only once each
%%%                    nonempty subword of #1
%%%                 2. replace all `/' by `:' in glyph codes
%%%                 3. simplified the command \special{!/M{...
%%%                 4. \mayaAddLigature, \mayaIInsitsToChange...
%%%                 5. \mayaGlyph{} (empty argument) yields a space
%%%                 6. (added 23.4.7) \mayaFont\a=font and \mayaFont\b=font
%%%                    create two links to the same font (now if a glyph is
%%%                    added to the font \a, then it is added automatically 
%%%                    to the font \b).
%%%                 7. mayaps.pro is separated from codex.mpf
%%%
%%%   version 0.20  1. Cancel the change 6 of v 0.19 by adding % before
%%%                    lines marked by "added 23.4.7". This change can be
%%%                    restored by removing the `%'.
%%%                 2. \maya@F -> \maya@FNum in the ligatures loading
%%%                    in the macro maya@aux (otherwise ligatures are not
%%%                    loaded from mpf's other the codex.mpf)
%%%                 3. Added \mayaGlobal(Un)define etc.
%%%                 4. Added \codexcol and \codexbw
%%%                 5. Glyphs with captions
%%%
%%%   version 0.21  1. Replace \codexcol and \codexbw by \red and \bw
%%%
%%%   version 0.22  1. Replace some \ifx with \if in \maya@GGN because
%%%                    otherwise !, ?, and : conflict with [french]{babel} 
%%%                 2. The default value of \mayahskip is changed
%%%                 3. \mayaGlyph[InLine]C*{CaptionText}{GlyphCode} and
%%%                    \mayaDebug#1  are added
%%%
%%%   version 0.23  1. Works correctly when mayaps.pro or *.mpf is not found
%%%                    by TeX, but is found by dvips
%%%                 2. Change of the fontsize in captions (macros
%%%                    \mayaCfive, \mayaCsix, ..., \mayaCsixteen
%%%                 3. Colors. The names \red,\bw,... are replaced by:
%%%                    \mayaRed,\mayaBW,\codexred,\codexbw,\gatesred,\gatesbw
%%%
%%%   version 0.24  1. Type 1 based fonts (see makefont, codex.mpf)
%%%
%%%   version 0.25  1. show instead of showglyph is used in fonts (see makefont)
%%%                 2. @ instead of & is used as the affix->central modifier
%%%                 3. See mayaps-d, 6.1."How a glyph is placed in the cartouche"
%%%                 4. Default value of \mayahspace is 1/10*\maya@xsize
%%%
%%%   version 0.26  1. The implementation of the ligature mechanism is rewritten 
%%%                    to avoid multiletter control sequences overflow
%%%                 2. make :!; non-active in captions (for [french]{babel}) 
%%%                 3. \long\def\maya and \long\def\mayaC. Now the argument of
%%%                    these macros may contain  many paragraphs.
%%%
%%%   version 0.27  1. Multi-color support : \mayaRGB{rrr ggg bbb}c
%%%                 2. Adjustment of the caption font to the glyph size
%%%                 3. The right quote ' acts as ! or | (symmetry modifyer)
%%%                 4. = instead of @ is used as the affix->central modifier
%%%
%%%   version 0.28  1. Dependence of the ratio w/h on the font. The command
%%%                    \mayaIInsistTo... is replaced by \mayaSetCartoucheAspect
%%%
%%%   version 0.29  1. Cancel \ifmaya@TmpFile and \ifmaya@TmpOpen
%%%
%%%   version 0.30  1. (in mayaps.pro) MayaColorDict is replaced by MayaDict
%%%                 2. \mayaIgnoreRGB
%%%                 3. \baselineskip=0pt in the captions
%%%                 4. \ifmayaCFixedWidth (fixed width of captions)
%%%                 %. ! is not a modifier
%%%
%%%   version 0.31  1. Default value of \mayahskip is 0 plus 1/30*\maya@ysize minus 
%%%                    1/30*\maya@ysize
%%%                 2. The default value of the \maya@depth is changed (see mayaps-d)
%%%                 3. Percent sign somewhere in \maya@Lc (to avoid extra spaces)
%%%                 4. \mayaRed and \mayaBW are put into another file  red89.tex
%%%                 5. The command \mpfmap (see file mpfmap.tex)
%%%
%%%   version 0.32  1. The parameters for \mayaC are made the same as for \maya
%%%                 2. \maya@depth changed to 1/2(h-1.5ex)+1/2h' - c (see mayaps-d)
%%%                 3. Default value of \mayavspace is 1/10*\maya@ysize
%%%                 4. Default value of \mayahspace is 1/30*\maya@ysize
%%%                 5. mayaGlyphInLineC parameters similar to mayaGlyphInLine
%%%                 6. Default value of \mayahskip is 0pt
%%%                 7. mayaps-d updated
%%%
%%%   version 0.33  1. '&' in glyph names is not accepted
%%%                 2. \mayaGlyphCC, \mayaGlyphCC*
%%%                 3. \maya@capt is extracted from \maya@GC
%%%                 4. see changes in mpfmap.tex
%%%                 5. Add `/showpage{}def' in mayaps.pro to fix \mayaAddGlyph
%%%                    (it did not work at least since v.0.21)
%%%                 6. \if\e\modif -> \ifx\e\modif in \maya@AG
%%%
%%%   version 0.34  1. \mayaImport[(L)(U)(R)(D)(S)]{newname}\fnt{oldname}A/C
%%%
%%%   version 0.35  1. Add substitution x->x in \mayaAddGlyph and \mayaImport
%%%                 2. \mayaRGB{...}c defines colors c and `c simultaneously.
%%%
%%%   version 0.36  1. Empty glyphs are fixed. One can use again ((((:).):).)
%%%                 2. Changes in the manual mayaps-d (mostly about colors)
%%%
%%%   version 0.37  1. Coloring of composed glyphs is allowed i.e. commands
%%%                    such as \maya{ (A.B:C)r.D } can be used now
%%%                 2. \mayaTOC
%%%                 3. We add %%VMusage 0 0 at the beginning of the file mayaps.tmp
%%%                    to aviod sectioning of the resulting ps file (the sectioning
%%%                    does not work correctly in TeX Live now). This is a cheating.
%%%                    Insert `\def\mayaNoVMtrick{}' before `\input mayaps'
%%%                    to cancel this cheating.
%%%                 4. see changes in mpfmap.tex
%%%
%%%   version 0.38  1. \maya@special is extracted from \mayaGlyph and it is
%%%                    called from \mayaGlyphInLine instead of \mayaGlyph
%%%                    As a consequence, \protect\maya, \protect\mayaGlyph
%%%                    and \protect\mayaGlyphInLine work properly in toc
%%%                 2. \mayaTOC = \protect\maya (in LaTeX) or \maya (in TeX)
%%%                 3. see changes in mpfmap.tex
%%%                 4. Message about VMtric is modified
%%%                 5. \mayaImport: the default [LURDS] is inherited from from
%%%                    the imported glyph
%%%                 6. The automatical preloading of the font codex can be
%%%                    disabled by inserting the command \def\mayaNoPreloadedFont{}
%%%                    before \input mayaps
%%%                 7. The absence of  red89.tex  is no longer fatal
%%%                 8. A bug in \mayaUndefine if fixed
%%%                 9. \mayaSize{y}:  y < 1pt is forbidden
%%%
%%%   version 0.39  1. Hypertext references in TOC in mayaps-d.tex
%%%                 2. Subection "Installation" in mayaps-d.tex
%%%
%%%   version 1.0   1. Substitutions are applied only to initial substrings or
%%%                    substrings preceded by one of .:/|'-+*?=](
%%%                 2. The "verbatim" bug (see mayaps-d v 0.38 Sect 8.1) is fixed
%%%                 3. \maya@a and \maya@b are extracted from \maya@aux
%%%                    which saved 20% of time of \output execution
%%%                 4. French spacing before `?' is suppressed in glyph captions
%%%                 5. Backslash is now allowed in eps files used in \mayaAddGlyph
%%%
%%%   version 1.1a  1. Infixation support: xxx<yyy>
%%%                    Most changes are in mayaps.pro
%%%                    The only change in the TeX-part is that `>' is added to the
%%%                    list of characters preceding a substitution
%%%
%%%   version 1.1b  1. Infixation support:
%%%                    1.1. The empty glyph bug is fixed;
%%%                    1.2. SetInfixScaleFactors (in mayaps.pro) is added;
%%%                    1.3. The corresponding changes in mayaps-d.tex
%%%
\catcode`\@=11
\def\maya@Message{\immediate\write16}
\maya@Message{This is `mayaps.tex' v 1.1b <March 4, 2017>}
%
%
\newread \maya@i
\newwrite\maya@o
%
\newdimen\maya@xsize
\newdimen\maya@ysize  \maya@ysize=12mm
\newdimen\maya@depth
\newdimen\maya@rsize
\newdimen\maya@tsize
\newdimen\maya@tmp
\newdimen\mayavcorrection \mayavcorrection=0pt
\newdimen\mayavspace
\newdimen\mayavspaceC
\newdimen\mayahspace
%
\newskip\mayahskip
%
\newcount\maya@intWidth
\newcount\maya@intHeight
\newcount\maya@ParDep
\newcount\maya@GlyphNum \maya@GlyphNum=0
\newcount\maya@n
\newcount\maya@FNum     \maya@FNum=0   % the number of maya fonts
\newcount\maya@FNumNF   \maya@FNumNF=0 % the number of Non Found fonts
\newcount\maya@F        \maya@F=0      % the current maya font
%
\newif\ifmaya@
\newif\ifmaya@header
\newif\ifmaya@iOK
\newif\ifmaya@oOK
\newif\ifmaya@c
\newif\ifmaya@AG
\newif\ifmaya@l
\newif\ifmayaCFW \mayaCFWtrue   % CFW = Caption of a Fixed Width

\newif\ifImport \Importfalse
%
\newtoks\maya@output
\newtoks\maya@gn
\newtoks\maya@lt
%
\global\let\maya@d=\maya@ParDep
\maya@Message{\ifnum\catcode`:=13 active\else nonactive\fi}
{\catcode`\:=12\gdef\maya@co{:}
 \catcode`\;=12\gdef\maya@qm{?}
 \catcode`\%=12\gdef\maya@pc{%}
}%
%
%  Names of macros defined in this file
%
%  \codex
%  \maya
%
%  \Maya@act
%  \maya@@a
%  \maya@@c
%  \maya@0@@aaa, \maya@0@@bbb, ..., \maya@1@@aaa, ...
%  \maya@0@0, \maya@0@1, ..., \maya@1@0, \maya@1@1, ...
%  \maya@0@GlyphNum, \maya@1@GlyphNum, ...
%  \maya@a
%  \maya@A
%  \maya@AC
%  \maya@AG
%  \maya@AGaux
%  \maya@AGwarn
%  \maya@aux
%  \maya@b
%  \maya@B
%  \maya@BC
%  \maya@c@aaa, \maya@c@bbb, ...
%  \maya@capt
%  \maya@CheckBP
%  \maya@CheckBPa
%  \maya@co
%  \maya@ff0, \maya@ff1, ...
%  \maya@fnaaa, \maya@fnbbb, ...                        % added 23.4.7
%  \maya@GC
%  \maya@GCC
%  \maya@GetGlyphNames
%  \maya@GGN
%  \maya@GGNc
%  \maya@GGNcA
%  \maya@GGNend
%  \maya@GGNSkipArg
%  \maya@GLC
%  \maya@Imp
%  \maya@io
%  \maya@L
%  \maya@L0@aaa, \maya@L0@bbb, ..., \maya@L1@aaa, ...
%  \maya@La
%  \maya@Lc
%  \maya@Lin
%  \maya@Lp
%  \maya@Lq
%  \maya@LT
%  \maya@LTa
%  \maya@LTb
%  \maya@LTc
%  \maya@LTd
%  \maya@LTe
%  \maya@LTf
%  \maya@Lz
%  \maya@Message
%  \maya@MsgNotFound
%  \maya@P
%  \maya@pc
%  \maya@qm
%  \maya@sA
%  \maya@sethskip
%  \maya@sethspace
%  \maya@setvspace
%  \maya@Slash
%  \maya@special
%  \maya@write
%  \maya@xsize@0, \maya@xsize@1, ...
%
%  \mayaAddGlyph
%  \mayaAddLigature
%  \mayaC
%  \mayaCFW
%  \mayaDebug
%  \mayaDefine
%  \mayaDeleteLigature
%  \mayaFont
%  \mayaGlobalAddLigature
%  \mayaGlobalDefine
%  \mayaGlobalDeleteLigature
%  \mayaGlobalUndefine
%  \mayaGlyph
%  \mayaGlyphC
%  \mayaGlyphCC
%  \mayaGlyphInLine
%  \mayaGlyphInLineC
%  \mayaIgnoreRGB
%  \mayaNoPreloadedFont
%  \mayaNoVMtrick
%  \mayaRGB
%  \mayaSetCatroucheSapect
%  \mayaSize
%  \mayaTOC
%  \mayaUndefine
%
%============================================================
%
%  The heart of the program: here we draw a composed glyph #1 using
%  the PostScript recursive procedure `parse' from `codex.mpf'
%  (`parse' is called from `E')
%
\def\mayaGlyph#1{%
  \hbox to \maya@xsize{\vbox to \maya@ysize{\vfil\maya@special{#1}}\hfil}}
%
\def\maya@special#1{\ifnum\maya@FNum=0
    \maya@Message{mayaps::WARNING: No Maya font is loaded. Glyph #1 is ignored.}%
  \else{\def\a{#1}\def\e{}\ifx\a\e\else
    \maya@intWidth=\maya@xsize   \maya@intHeight=\maya@ysize
%   the next line is just for saving 4 bytes per glyph in the ps file
    \divide\maya@intWidth by 100 \divide\maya@intHeight by 100
    \maya@Slash{#1}\maya@lt={}\let\E=\expandafter\maya@true%  See TeXbook, p. 374:
    \E\E\E\maya@L\E{\the\maya@gn}\E\E\E\maya@CheckBP\E{\the\maya@lt}\ifmaya@
       \E\E\E\maya@GetGlyphNames\E{\the\maya@lt}%
       \special{"M(\the\maya@lt)\the\maya@intWidth\space
           \the\maya@intHeight\space \the\maya@F\space E}%
    \else\special{"M(bad "( )")\the\maya@intWidth\space
           \the\maya@intHeight\space \the\maya@F\space E}\fi\fi}\fi}
%========================================================
%
%  Set  \maya@ysize = #1, set  \maya@xsize  proportionally,
%  and set the corresponding default values of \mayavspace,\mayahskip
%  and set the font in captions
%
\def\mayaSize#1{% 
    \maya@rsize=1in \maya@ysize=#1
    \maya@tmp=1pt \ifnum\maya@ysize<\maya@tmp
      \maya@Message{mayaps::WARNING: minimal Maya font size is 1pt}
      \maya@ysize=1pt
    \fi
    \ifnum\maya@FNum>0
      \maya@tsize=\csname maya@xsize@\the\maya@F\endcsname in
    \else\maya@tsize=1.53333in\fi
%
%  Here we adapt Tom Rokicki's code from `epsf.tex' to compute  x = ty/r
%  (rather than  y = rx/t  as is written in a comment in epsf.tex v 2.7k).
%  He claims that the error is at most 1/4000 pt
%
    \maya@tmp=\maya@tsize \divide\maya@tmp\maya@rsize
    \maya@xsize=\maya@ysize \multiply\maya@xsize\maya@tmp
    \multiply\maya@tmp\maya@rsize \advance\maya@tsize-\maya@tmp
    \maya@tmp=\maya@ysize
    {%                     start a new group to hide \loop
     \loop \advance\maya@tsize\maya@tsize \divide\maya@tmp 2
     \ifnum \maya@tmp>0
        \ifnum \maya@tsize<\maya@rsize
        \else
           \advance\maya@tsize-\maya@rsize \advance\maya@xsize\maya@tmp
        \fi
     \repeat
     \global\maya@tmp=\maya@xsize
    }%
    \maya@xsize=\maya@tmp
%                                  Adjustment of the caption font size
    \maya@sethskip \maya@setvspace \maya@sethspace
    \maya@tmp=19mm
    \ifnum\maya@ysize<\maya@tmp \maya@tmp=11mm
      \ifnum\maya@ysize<\maya@tmp \maya@tmp=7mm
        \ifnum\maya@ysize<\maya@tmp \maya@tmp=5mm
          \ifnum\maya@ysize<\maya@tmp\mayaCfive\else\mayaCsix\fi
        \else\maya@tmp=9mm
          \ifnum\maya@ysize<\maya@tmp\mayaCseven\else\mayaCeight\fi\fi
      \else\maya@tmp=15mm
        \ifnum\maya@ysize<\maya@tmp \maya@tmp=13mm
          \ifnum\maya@ysize<\maya@tmp\mayaCnine\else\mayaCten\fi
        \else\maya@tmp=17mm
          \ifnum\maya@ysize<\maya@tmp\mayaCeleven\else\mayaCtwelve\fi\fi\fi
    \else\maya@tmp=27mm
      \ifnum\maya@ysize<\maya@tmp \maya@tmp=23mm
        \ifnum\maya@ysize<\maya@tmp \maya@tmp=21mm
          \ifnum\maya@ysize<\maya@tmp\mayaCthirteen\else\mayaCfourteen\fi
        \else\maya@tmp=25mm
          \ifnum\maya@ysize<\maya@tmp\mayaCfifteen\else\mayaCsixteen\fi\fi
      \else\maya@tmp=29mm
        \ifnum\maya@ysize<\maya@tmp\mayaCseventeen\else\mayaCeighteen\fi\fi\fi
}
%
\def\mayaSetCartoucheAspect#1{%
  \expandafter\edef\csname maya@xsize@\the\maya@F\endcsname{#1}%
  \mayaSize{\maya@ysize}}
%
\def\maya@sethskip{\mayahskip=0pt}
\def\maya@setvspace{%
   \mayavspace = \maya@ysize \divide\mayavspace by 10}
\def\maya@sethspace{%
   \mayahspace = \maya@ysize \divide\mayahspace by 30}
%
%========================================================
% The main entry from the exterior.
% Calling sequence:
%
%   \maya{GlyphCode1 GlyphCode2 ... }
%
% where each of  GlyphCode1, GlyphCode2, ...  is a code of
% a composed glyph, e.g. 
% \maya{ 123 234:345 (123.[R]234):345.-456 }
%
\def\mayaGlyphInLine#1{\maya@depth=\maya@ysize
    \advance\maya@depth-1.5ex \divide\maya@depth by 2
    \maya@tsize=\maya@ysize \advance\maya@tsize by \mayavspace
    \maya@tmp=\mayavspace \divide\maya@tmp by 2
    \advance\maya@depth\maya@tmp\advance\maya@depth-\mayavcorrection 
    \maya@tmp=\maya@xsize  \advance\maya@tmp by \mayahspace
    \ifvmode\hskip 0pt\fi
    \lower\maya@depth\hbox to \maya@tmp{%
        \divide\mayahspace by 2 \divide\mayavspace by 2
        \hskip\mayahspace\vbox to \maya@tsize{%
            \vfil\maya@special{#1}\vskip\mayavspace}\hfil}}
%
%   d = y; d = y - 1.5 ex; d = (y - 1.5 ex)/2;
%   t = y; t = y + v;
%   tmp = v; tmp = v/2
%   d = (y + v - 1.5 ex)/2; d = (y + v - 1.5 ex)/2 - vcorr
%   tmp = x + h
%
\long\def\maya#1{\maya@A{ #1 @ @ }}
\long\def\maya@A#1{\maya@B #1 @]}
\long\def\maya@B #1#2 #3#4 @]{\def\a@{@}\def\a{#1}\ifx\a\a@\else\def\p{\par}
  \ifx\a\p\par\maya@B #2 #3#4 @]%
  \else\mayaGlyphInLine{#1#2}\def\b{#3}\ifx\b\a@\else%\space
       \hskip\mayahskip\maya@B #3#4 @]\fi\fi\fi}
%
%====================================================================
% To use with LaTeX in the table of content (toc) and such.
%
\def\mayaTOC#1{\ifx\documentstyle\mayaUndefinedMacro\maya{#1}
    \else\protect\maya{#1}\fi}
%
%========================================================
% Versions of \mayaGliph, \mayaGlyphInLine, and \maya
% with captions under the glyphs
%
\mayavspaceC=3pt \def\maya@@s{*}
\def\mayaGlyphC#1{\def\maya@@a{#1}%
  \ifx\maya@@a\maya@@s\expandafter\maya@GC\else\maya@GC{#1}{#1}\fi}
\def\maya@GC#1#2{\hbox to \maya@xsize{\baselineskip=0pt
  \vbox{\vbox{\mayaGlyph{#2}}\vskip\mayavspaceC\mayaCaptionFont
    \maya@capt{#1}}}}%
{\catcode`\:=13 \catcode`\?=13           % against [french]{babel}
\gdef\maya@capt#1{%
  \vbox{\hbox to \maya@xsize{\hss\def:{\maya@co}\def?{\maya@qm}%
    \setbox1=\hbox{#1}%
    \ifnum\wd1>\maya@xsize\ifmayaCFW\maya@n=\wd1\divide\maya@n by 100
      \maya@tsize=\maya@xsize\advance\maya@tsize-\wd1
      \maya@intWidth=\maya@xsize\divide\maya@intWidth by 100
      \special{ps:\the\maya@intWidth\space\the\maya@n\space m@Ya}%
      \box1\hskip\maya@tsize\special{ps:grestore}%
    \else\box1\fi\else\box1\fi\hss}}}
}% end of catcode changes
\def\mayaGlyphInLineC#1{\def\maya@@a{#1}%
  \ifx\maya@@a\maya@@s\expandafter\maya@GLC\else\maya@GLC{#1}{#1}\fi}
\def\maya@GLC#1#2{%
    {\mayaCaptionFont\setbox1=\hbox{\strut}\global\maya@tmp=\ht1
        \global\advance\maya@tmp\dp1}%
    \advance\maya@tmp by \mayavspaceC
    \maya@depth=\maya@ysize  \advance\maya@depth-1.5ex
    \advance\maya@depth by \maya@tmp
    \divide\maya@depth by 2
    \maya@tsize=\maya@ysize \advance\maya@tsize by \mayavspace
    \advance\maya@tsize by \maya@tmp
    \maya@tmp=\mayavspace \divide\maya@tmp by 2
    \advance\maya@depth by \maya@tmp
    \advance\maya@depth-\mayavcorrection 
    \maya@tmp=\maya@xsize  \advance\maya@tmp by \mayahspace
    \ifvmode\hskip 0pt\fi
    \lower\maya@depth\hbox to \maya@tmp{\hss\vbox to \maya@tsize
        {\hsize=\maya@xsize\vfil\noindent\maya@GC{#1}{#2}\vfil}\hss}}
%
\long\def\mayaC#1{\maya@AC{ #1 @ @ }}
\long\def\maya@AC#1{\maya@BC #1 @]}
\long\def\maya@BC #1#2 #3#4 @]{\def\a@{@}\def\a{#1}\ifx\a\a@\else\def\p{\par}
  \ifx\a\p\par\maya@BC #2 #3#4 @]%
  \else\mayaGlyphInLineC{#1#2}\def\b{#3}\ifx\b\a@\else%\space
       \hskip\mayahskip\maya@BC #3#4 @]\fi\fi\fi}
%=============================================================
% Versions of \mayaGliph with double captions under the glyphs
%
\def\mayaGlyphCC#1{\def\maya@@a{#1}%
  \ifx\maya@@a\maya@@s\expandafter\maya@GCC
  \else\maya@GCC{#1}{{\let\tt=\mayaCaptionFont\mayaDebug{#1}}}{#1}\fi}
\def\maya@GCC#1#2#3{\hbox to \maya@xsize{\baselineskip=0pt
  \vbox{\vbox{\mayaGlyph{#3}}\vskip\mayavspaceC\mayaCaptionFont
    \maya@capt{#1}\vskip\mayavspaceC\maya@capt{#2}}}}%
%==============================================================
%
%  Changing font in \mayaC.
%
\def\mayaCeighteen{\expandafter\ifx\csname maya@Cf18\endcsname\relax
    \global\expandafter\def\csname maya@Cf18\endcsname{}\global\font
    \maya@eighteen=cmtt12 at 18pt\fi\let\mayaCaptionFont=\maya@eighteen}
\def\mayaCseventeen{\expandafter\ifx\csname maya@Cf17\endcsname\relax
    \global\expandafter\def\csname maya@Cf17\endcsname{}\global\font
    \maya@seventeen=cmtt12 at 17pt\fi\let\mayaCaptionFont=\maya@seventeen}
\def\mayaCseventeen{\expandafter\ifx\csname maya@Cf17\endcsname\relax
    \global\expandafter\def\csname maya@Cf17\endcsname{}\global\font
    \maya@seventeen=cmtt12 at 17pt\fi\let\mayaCaptionFont=\maya@seventeen}
\def\mayaCsixteen{\expandafter\ifx\csname maya@Cf16\endcsname\relax
    \global\expandafter\def\csname maya@Cf16\endcsname{}\global\font
    \maya@sixteen=cmtt12 at 16pt\fi\let\mayaCaptionFont=\maya@sixteen}
\def\mayaCfifteen{\expandafter\ifx\csname maya@Cf15\endcsname\relax
    \global\expandafter\def\csname maya@Cf15\endcsname{}\global\font
    \maya@fifteen=cmtt12 at 15pt\fi\let\mayaCaptionFont=\maya@fifteen}
\def\mayaCfourteen{\expandafter\ifx\csname maya@Cf14\endcsname\relax
    \global\expandafter\def\csname maya@Cf14\endcsname{}\global\font
    \maya@fourteen=cmtt12 at 14pt\fi\let\mayaCaptionFont=\maya@fourteen}
\def\mayaCthirteen{\expandafter\ifx\csname maya@Cf13\endcsname\relax
    \global\expandafter\def\csname maya@Cf13\endcsname{}\global\font
    \maya@thirteen=cmtt12 at 13pt\fi\let\mayaCaptionFont=\maya@thirteen}
\def\mayaCtwelve{\expandafter\ifx\csname maya@Cf12\endcsname\relax
    \global\expandafter\def\csname maya@Cf12\endcsname{}\global\font
    \maya@twelve=cmtt12\fi\let\mayaCaptionFont=\maya@twelve}
\def\mayaCeleven{\expandafter\ifx\csname maya@Cf11\endcsname\relax
    \global\expandafter\def\csname maya@Cf11\endcsname{}\global\font
    \maya@eleven=cmtt10 at 11pt\fi\let\mayaCaptionFont=\maya@eleven}
\def\mayaCten{\expandafter\ifx\csname maya@Cf10\endcsname\relax
    \global\expandafter\def\csname maya@Cf10\endcsname{}\global\font
    \maya@ten=cmtt10 \fi\let\mayaCaptionFont=\maya@ten}
\def\mayaCnine{\expandafter\ifx\csname maya@Cf9\endcsname\relax
    \global\expandafter\def\csname maya@Cf9\endcsname{}\global\font
    \maya@nine=cmtt9 \fi\let\mayaCaptionFont=\maya@nine}
\def\mayaCeight{\expandafter\ifx\csname maya@Cf8\endcsname\relax
    \global\expandafter\def\csname maya@Cf8\endcsname{}\global\font
    \maya@eight=cmtt8 \fi\let\mayaCaptionFont=\maya@eight}
\def\mayaCseven{\expandafter\ifx\csname maya@Cf7\endcsname\relax
    \global\expandafter\def\csname maya@Cf7\endcsname{}\global\font
    \maya@seven=cmtt8 at 7pt\fi\let\mayaCaptionFont=\maya@seven}
\def\mayaCsix{\expandafter\ifx\csname maya@Cf6\endcsname\relax
    \global\expandafter\def\csname maya@Cf6\endcsname{}\global\font
    \maya@six=cmtt8 at 6pt\fi\let\mayaCaptionFont=\maya@six}
\def\mayaCfive{\expandafter\ifx\csname maya@Cf5\endcsname\relax
    \global\expandafter\def\csname maya@Cf5\endcsname{}\global\font
    \maya@five=cmtt8 at 5pt \fi\let\mayaCaptionFont=\maya@five}
%
\mayaCseven
%
%==============================================================
%
%  Replace in #1 all `/' by `:'. Write the result to \maya@gn
%
\def\maya@Slash#1{\maya@gn={}\def\a{#1}\def\e{}\ifx\a\e\else\maya@sA#1 @]\fi}
\def\maya@sA#1#2 @]{\def\a{#1}\def\b{#2}\def\s{/}\def\e{}%
    \ifx\a\s\maya@gn=\expandafter{\the\maya@gn  :}%
    \else   \maya@gn=\expandafter{\the\maya@gn #1}\fi
    \ifx\b\e\else\maya@sA#2 @]\fi}
%
%==============================================================
%
% Check if parentheses are balanced
%
\def\maya@CheckBP#1{\ifImport\maya@Message{[#1]}\fi\def\e{}\def\a{#1}\ifx\a\e\maya@true
    \else \maya@ParDep=0\maya@CheckBPa#1 @]\fi}
\def\maya@CheckBPa#1#2 @]{\def\a{#1}\def\L{(}
    \ifx\a\L\advance\maya@ParDep by 1 %
    \else\def\R{)}\ifx\a\R\advance\maya@ParDep by -1\fi\fi
    \ifnum0>\maya@ParDep\maya@false
    \else\def\e{}\def\a{#2}%
        \ifx\a\e\ifnum0=\maya@ParDep\maya@true\else\maya@false\fi
        \else\maya@CheckBPa#2 @]\fi\fi}
%
%================================================================
% This section (till the next line "%========") is devoted to
% selecting only those glyphs from the file  codex.mpf  which are
% really used in the document.
%
% Here we extract glyph names from #1.
% For each extracted name xxx we check if a macro \maya@f@@xxx is defined
% where f is the current Maya font (the value of the counter \maya@F).
% If it isn't then we define it and also \def\maya@f@n{xxx} where 
% n is the current value of the counter \maya@f@GlyphNum which takes
% successive values 0,1,2,... starting from the beginning of each page
%
\def\maya@GetGlyphNames#1{\E\ifx\csname maya@ff\the\maya@F\endcsname\relax
  \else\def\a{#1}\ifx\e\a\else\maya@gn={}\maya@n=0\maya@ltrue\maya@GGN#1 \fi\fi}
\def\maya@GGN#1#2 {\def\c{\noexpand#1}\def\a{#2}
    \if\c\noexpand[\ifx\a\e\else\maya@GGNSkipArg#2 \fi
    \else\maya@false
        \if\c\noexpand<\maya@true \else
        \if\c\noexpand>\maya@true \else
        \if\c\noexpand-\maya@true \else
        \if\c\noexpand|\maya@true \else
        \if\c\noexpand=\maya@true \else
        \if\c\noexpand+\maya@true \else
        \if\c\noexpand?\maya@true \else
        \if\c\noexpand*\maya@true \else
        \if\c\noexpand'\maya@true \else
        \if\c\noexpand(\maya@true \else
        \if\c\noexpand)\maya@true\maya@lfalse \else
        \if\c\noexpand/\maya@true\maya@ltrue \else
        \if\c\noexpand.\maya@true\maya@ltrue \else
        \if\c\noexpand:\maya@true\maya@ltrue \fi\fi\fi\fi\fi\fi\fi\fi\fi\fi\fi\fi\fi\fi
        \ifmaya@ \maya@GGNend \ifx\a\e\else\maya@GGN#2 \fi
        \else\ifmaya@l\advance\maya@n by 1
            \maya@gn=\E{\the\maya@gn #1}\fi     % TeXbook, p.373
            \ifx\a\e \maya@GGNend \else\maya@GGN#2 \fi\fi\fi}
\def\maya@GGNSkipArg#1#2 {\def\a{#2}\ifx\a\e\else\def\c{\noexpand#1}
    \if\c\noexpand]\maya@GGN#2 \else\maya@GGNSkipArg#2 \fi\fi}
\def\OLDmaya@GGNSkipArg#1#2 {\def\a{#2}\ifx\a\e\else\def\c{#1}
    \def\rb{]}\if\c\rb\maya@GGN#2 \else\maya@GGNSkipArg#2 \fi\fi}
\def\maya@GGNend{\ifnum\maya@n>0
%
% (Added in v 0.27 for the multicolor support)
% First, we remove the color indicator from the end of \maya@gn
%
  \E\E\E\maya@GGNc\E{\the\maya@gn}%
%
%   Here we must check if a macro \maya@f@@xxx is defined
%   where xxx means the value of the token list register \maya@gn
%   and f is the value of the counter \maya@F (see TeXbook, Ex 7.7)
%
  \E\ifx\csname maya@\the\maya@F @@\the\maya@gn\endcsname\relax
    \global\maya@GlyphNum=\E\csname maya@\the\maya@F
      @GlyphNum\endcsname\relax% It does not work without \relax. Why???... 
    \E\xdef\csname maya@\the\maya@F
      @\the\maya@GlyphNum\endcsname{\the\maya@gn}
    \E\xdef\csname maya@\the\maya@F @@\the\maya@gn\endcsname{}%
    \global\advance\maya@GlyphNum by 1
    \E\xdef\csname maya@\the\maya@F @GlyphNum\endcsname{%
      \the\maya@GlyphNum}\fi
  \maya@gn={}\maya@n=0 \fi}
\def\maya@GGNc#1{\maya@gn={}\maya@GGNcA#1 @]}
\def\maya@GGNcA#1#2 @]{\def\b{#2}\maya@gn=\E{\the\maya@gn #1}%
  \ifx\b\e\else\E\ifx\csname maya@c@\b\endcsname\relax\maya@GGNcA#2 @]\fi\fi}
%
%
%   The following macro \maya@io reads the input file line by line
%   and does the following.
%   If the flag \maya@header is true then it copies to the output
%   file everything (except the comments) before the first line starting
%   with `%@' and then stops. For each line starting with `%L@ xxx yyy',
%   the macro \maya@fL@xxx with the extension yyy is defined (it will be
%   used for ligatures). Here f is the value of \maya@FNum
%
%   If the flag \maya@header is false then it looks for lines
%   starting with `%@ xxx' where xxx is the extension of one of macros
%   \maya@f@0, \maya@f@1, ... \maya@f@nnn (nnn is the value of the macro 
%   \maya@f@GlyphNum and f is the current Maya font, i.e., the value of
%   the counter \maya@F), and copies all further lines till the next line 
%   starting with `%@'.
%
%   Here we again adapt a code from `epsf.tex' 
%   (comments %" are also copied from there).
%
\def\maya@io{%
        {%                      %"start a group to contain catcode changes
            \def\e{\par}
            \maya@iOKtrue         %"true while we are looping
            \ifmaya@header\maya@oOKtrue\else\maya@oOKfalse \fi
            %" Make all special characters, except space, to be of type
            %" `other' so we process the file in almost verbatim mode
            %" (TeXbook, p. 344).
            \chardef\other=12
            \def\do##1{\catcode`##1=\other}% space \ { } $ & # ^ ^K ^A % ~
            \dospecials
            \catcode`\ =10 \catcode`\@=11
            \catcode`\^^M=5    % catcodes of ^M < > are changed for the case
            \catcode`\<=\other % when the end of page appears between latex's
            \catcode`\>=\other % \begin{verbatim} and \end{verbatim}
            \loop                 %"reading lines from the mpf file
                \read\maya@i to \maya@line 
                \ifeof\maya@i         %"then no more input
                    \maya@iOKfalse    %"so set completion flag
                \else                 %"otherwise process one line
                    \ifx\e\maya@line \else   % to avoid `\par' in ps file
                        \expandafter\maya@aux\maya@line . . . . . \\%
                        \ifmaya@oOK
                            \ifmaya@c\else
                                \immediate\write\maya@o{\maya@line}%
                            \fi
                        \fi
                    \fi
                \fi
            \ifmaya@iOK
            \repeat
        }%                      %"end catcode changes
        \closein\maya@i
}
%
{\catcode`\%=12 \global\let\maya@P=%}
%
% this macro is used in \maya@io
%
% The extraction of \maya@a and \maya@b from \maya@aux saved 20% of time
% on the execution of \output
%
\long\def\maya@aux#1#2 #3 #4 #5 \\{\maya@cfalse\ifx#1\maya@P\maya@a{#2}{#3}{#4}\fi}
\long\def\maya@a#1#2#3{\maya@ctrue\def\a{@}\def\b{#1}%
        \ifx\a\b\maya@b{#2}%
        \else                        % %L or %W line
            \ifmaya@header\def\a{L@}%          %L  initialize the ligature table
                \ifx\a\b\mayaGlobalDefine{#2}{#3}%
                \else\def\a{W@}\ifx\a\b        %W  read the cartouche width
                    \expandafter\xdef\csname maya@xsize@\the\maya@F\endcsname{#2}
                \fi\fi
            \fi
        \fi
}
\def\maya@b#1{%
        \maya@oOKfalse
        \ifmaya@header\maya@iOKfalse
        \else \ifnum\maya@GlyphNum>0 \maya@oOKtrue\fi
        \fi
        \ifmaya@oOK
                \def\a{.}\def\b{#1}\maya@oOKfalse
                \ifx\a\b\else\maya@n=0
                    {\loop             % start a group because of embedded loops
                    \ifnum\maya@n<\maya@GlyphNum
                        \ifmaya@oOK
                            \maya@d=\maya@n \advance\maya@d-1
                            \expandafter\xdef%
                                \csname maya@\the\maya@F @\the\maya@d\endcsname%
                                {\csname maya@\the\maya@F @\the\maya@n\endcsname}%
                        \else
                            \expandafter\ifx%
                                \csname maya@\the\maya@F @\the\maya@n\endcsname\b
                                \global\maya@oOKtrue
                            \fi
                        \fi
                        \advance\maya@n by 1
                    \repeat}%          % end of the embedded loop
                    \ifmaya@oOK\global\advance\maya@GlyphNum-1\fi
                \fi
        \fi
}
%===================================================================
%                                                 Multifont support
%
%   This macro works more or less like TeX's macro \font:
%     \mayaFont\cs=foo
%   where  \cs  is a control sequence and  foo.mpf  is a maya font file.
%   It creates the macro \maya@ff<fn> (<fn> = 0,1,...) -> foo.mpf
%   It defines also a macro \cs whose action is switching the counter
%   \maya@F to the font number associated to \cs
%
\def\mayaFont#1=#2 {%
% \expandafter\ifx\csname maya@fn#2\endcsname\relax              % added 23.4.7
%    \expandafter\xdef\csname maya@fn#2\endcsname{\the\maya@FNum}% added 23.4.7
    \xdef#1{\maya@F=\the\maya@FNum\noexpand\mayaSize\noexpand{\noexpand\maya@ysize}}%
    \expandafter\xdef\csname maya@ff\the\maya@FNum\endcsname{#2.mpf}%
    \expandafter\xdef\csname maya@\the\maya@FNum @GlyphNum\endcsname{0}%
    \expandafter\xdef\csname maya@L\the\maya@FNum @\endcsname{0 1}%
    \expandafter\xdef\csname maya@xsize@\the\maya@FNum\endcsname{1.53333}%
    \openin\maya@i=#2.mpf
    \ifeof\maya@i
        \maya@MsgNotFound{#2.mpf}\special{header=#2.mpf}%
        \global\expandafter\let\csname maya@ff\the\maya@FNum\endcsname=\relax%
        \maya@Message{mayaFont::WARNING: No ligatures are loaded in font `#2'.}%
        \maya@Message{\space\space\space\space\space\noexpand\mayaAddGlyph
              \space commands will be ignored for this font.}%
        \maya@Message{}\immediate\write\maya@o%
              {userdict begin MayaDict begin SkipFont end end}%
    \else
        \immediate\write\maya@o{userdict begin MayaDict begin tmpini end end}%
        \maya@headertrue
        \maya@n=\maya@F\maya@F=\maya@FNum\maya@FNum=\maya@n
        \maya@io
        \maya@n=\maya@F\maya@F=\maya@FNum\maya@FNum=\maya@n
        \maya@headerfalse
        \immediate\write\maya@o{userdict begin MayaDict begin tmpend end end}%
    \fi
    \global\advance\maya@FNum by 1%
% \else\xdef#1{\maya@F=\csname maya@fn#2\endcsname}\fi%           added 23.4.7
}
%==================================================================
%                                                Multi-Color support
%
\def\mayaRGB#1#2{%\mayaGlobalDefine{#2}{#2}%
  \expandafter\xdef\csname maya@c@#2\endcsname{1}%
  \expandafter\xdef\csname maya@c@`#2\endcsname{1}\special{%
  !userdict begin MayaDict begin ColorDict begin/c#2{#1}def/c`#2{#1}def end end end}}
\expandafter\def\csname maya@c@`\endcsname{1}%
%
\def\mayaIgnoreRGB{\special{!userdict begin MayaDict begin/RGB{false}def end end}}
%==================================================================
%                                                 Ligature support
%  Example:  after the command
%
%            \mayaAddLigature{123.456}{789}
%
%            `123.456' will be replaced with `789' in each glyph code
%            of the font which was current when \mayaAddLigature was executed
%
%
% check if a character #1 belongs to a string #2. The result: \ifmaya@
\def\maya@Lin#1#2{\def\a{#2}\def\e{}%
  \ifx\a\e\global\maya@false\else\maya@LinA#1#2 @]\fi\relax}
\def\maya@LinA#1#2#3 @]{\def\a{#1}\def\b{#2}%
  \if\a\b\global\maya@true\else\maya@Lin#1{#3}\fi\relax}
%\def\In#1#2{\maya@Lin#1{#2}\ifmaya@l True \else False \fi\relax}
%
%   The macro \maya@LT creates/updates the ligature tree by adding
%   the ligature  \maya@gn -> #1 to the current font <fn>.
%   The ligature tree means the following. For any ligature A->B
%   are defined macros \maya@L<fn>@<W> for all intial subwords W of A
%   including the empty subword and A itself. The value (i.e. expansion)
%   of such macro is a string of the form "0S 1R" where R is as in
%   W->R if such ligature exists and R is empty if there is no
%   ligature of the form W->...; S is a string of all characters c such
%   that Wc is an initial subword of the left hand side of some ligature.
%
%   Before calling this macro, \maya@def should be defined
%   either by  \let\maya@def=edef  or by  \let\maya@def=xdef
%
\def\maya@LT#1{%
  \expandafter\ifx\csname maya@L\the\maya@F @\the\maya@gn\endcsname\relax
    \expandafter\maya@def\csname maya@L\the\maya@F @\the\maya@gn\endcsname{0 1#1}%
  \else\edef\Maya@act{\noexpand\maya@LTe{#1
         \csname maya@L\the\maya@F @\the\maya@gn\endcsname}}\Maya@act\fi
  \maya@lt={}\edef\Maya@act{\noexpand\maya@LTa{\the\maya@gn}}\Maya@act}
\def\maya@LTa#1{\maya@LTb#1 @]}
\def\maya@LTb#1#2 @]{%
  \expandafter\ifx\csname maya@L\the\maya@F @\the\maya@lt\endcsname\relax
     \expandafter\maya@def\csname maya@L\the\maya@F @\the\maya@lt\endcsname{0#1 1}%
  \else\edef\Maya@act{\noexpand\maya@LTc{%
      \csname maya@L\the\maya@F @\the\maya@lt\endcsname\space #1}}\Maya@act\fi
  {\def\b{#2}\def\e{}\ifx\b\e\global\maya@lfalse\else\global\maya@ltrue\fi}%
  \ifmaya@l\edef\Maya@act{\noexpand\maya@lt={\the\maya@lt #1}}\Maya@act
     \maya@LTb#2 @]\fi}
\def\maya@LTc#1{\maya@LTd#1 @]}
\def\maya@LTd0#1 1#2 #3 @]{{\maya@Lin#3{#1}}%
  \expandafter\maya@def\csname maya@L\the\maya@F @\the\maya@lt\endcsname{%
     0#1\ifmaya@\else#3\fi\space 1#2}}
\def\maya@LTe#1{\maya@LTf#1 @]}
\def\maya@LTf#1 #2 #3 @]{%
  \expandafter\maya@def\csname maya@L\the\maya@F @\the\maya@gn\endcsname{#2 1#1}}
%
%   The macro  \maya@L  applies the ligature table of the current
%   font to the argumemt `#1' and appends the result to the token list
%   register \maya@lt
%   The substitution is applied only when maya@true. When \maya@L is called
%   recursively, this flag is set on only after one of ](./:*+-='?|<
%
\def\maya@L#1{\def\e{}\def\a{#1}\let\NE=\noexpand\ifx\a\e\maya@ltrue\else
    \ifmaya@\maya@Lc0 #1 @]\ifmaya@l\else\maya@La#1 @]\fi\else\maya@La#1 @]\fi\fi}
\def\maya@La#1#2 @]{\global\maya@lt=\expandafter{\the\maya@lt #1}%
  \maya@Lin{#1}{](./:*+-='?|<}\maya@L{#2}}
\def\maya@Lc0#1 #2#3 @]{\edef\t{\csname maya@L\the\maya@F @#1\endcsname}
  \edef\act{\NE\maya@Lp\t\space @]}\act\edef\act{\NE\maya@Lin#2\NE{\p}}\act
  \ifmaya@\def\a{#3}%
    \ifx\a\e\maya@Lz{#1#2}{#3}%
    \else\maya@Lc0#1#2 #3 @]\ifmaya@l\else\maya@Lz{#1}{#2#3}\fi\fi
  \else\maya@Lz{#1}{#2#3}\fi}
\def\maya@Lp0#1 1#2 @]{\def\p{#1}}\def\maya@Lq0#1 1#2 @]{\def\q{#2}}
\def\maya@Lz#1#2{\edef\t{\csname maya@L\the\maya@F @#1\endcsname}%
  \edef\act{\NE\maya@Lq\t\space @]}\act
  \ifx\e\q\maya@lfalse\relax
  \else\edef\act{\NE\global\NE\maya@lt=\noexpand{\the\maya@lt\q}}\act
     \maya@false\maya@L{#2}\fi\relax}
%
\def\mayaDefine#1#2{{\maya@Slash{#1}%
  \edef\act{\noexpand\global\noexpand\maya@gn={\the\maya@gn}}\act}%
  \let\maya@def=\edef\maya@LT{#2}}
%
\def\mayaGlobalDefine#1#2{{\maya@Slash{#1}\let\maya@def=\xdef\maya@LT{#2}}}
%
\def\mayaGlobalUndefine#1{{\maya@Slash{#1}\let\E=\expandafter
  \E\ifx\csname maya@L\the\maya@F @\the\maya@gn\endcsname\relax\else
    \edef\t{\csname maya@L\the\maya@F @\the\maya@gn\endcsname}%
    \edef\act{\noexpand\maya@Lp\t\space @]}\act
    \E\xdef\csname maya@L\the\maya@F @\the\maya@gn\endcsname{0\p\space 1}\fi}}
%
\def\mayaUndefine#1{{\maya@Slash{#1}%
  \edef\act{\noexpand\global\noexpand\maya@gn={\the\maya@gn}}\act}%
  \expandafter\ifx\csname maya@L\the\maya@F @\the\maya@gn\endcsname\relax\else
    {\edef\t{\csname maya@L\the\maya@F @\the\maya@gn\endcsname}%
     \edef\act{\noexpand\maya@Lp\t\space @]}\act
     \edef\act{\noexpand\global\noexpand\maya@lt={\p}}\act}%
     \expandafter\edef\csname maya@L\the\maya@F
             @\the\maya@gn\endcsname{0\the\maya@lt\space 1}\fi}
%
\let\mayaAddLigature=\mayaDefine
\let\mayaGlobalAddLigature=\mayaGlobalDefine
\let\mayaDeleteLigature=\mayaUndefine
\let\mayaGlobalDeleteLigature=\mayaGlobalUndefine
%
\def\mayaDebug#1{{%
    \maya@Slash{#1}\maya@lt={}\let\E=\expandafter% copied from \mayaGlyph
    \setbox1=\hbox{\maya@true\E\E\E\maya@L\E{\the\maya@gn}}% box to through away
    \tt\the\maya@lt}}
%===================================================================
%
% Open file  mayaps.pro  and copy the header into  mayaps.tmp
%
\def\maya@MsgNotFound#1{\maya@Message%
    {Cannot open `#1'. Hope it will be found by dvips...}}%
\immediate\openout\maya@o=mayaps.tmp
\ifx\mayaNoVMtrick\mayaUndefinedMacro\relax
  \maya@Message{Virtual Memory trick is used in mayaps.tmp: \maya@pc\maya@pc VMusage 0 0}
  \immediate\write\maya@o{\maya@pc\maya@pc VMusage\maya@co\space 0\space 0}%
\fi
\openin\maya@i=mayaps.pro
\ifeof\maya@i \maya@MsgNotFound{mayaps.pro}\special{header=mayaps.pro}%
\else
    \maya@headertrue
    \maya@io
    \maya@headerfalse
\fi
\maya@AGtrue
%===============================================================
%  dvips inserts this header when userdict is current
%  so userdict gets MayaDict and MayaDict gets E, parse, etc.
%
\special{header=mayaps.tmp}
%
%  Let's hope that all future versions of dvips will insert
%  this line after the texc.pro ...
%
\special{!/M{userdict begin MayaDict begin gsave}def}
%                                             dvips writes this to SDict
%
%======================================================================
% Addition to the output routine (the code which is executed at the end
% of every page).
%
\maya@output={%\maya@Message{mayaps: GlyphNum=\the\maya@GlyphNum}%
  \ifnum\maya@FNum>0
    \maya@F=0
    \loop
        \maya@GlyphNum=\expandafter%
            \csname maya@\the\maya@F @GlyphNum\endcsname\relax
        \ifnum\maya@GlyphNum>0
            \openin\maya@i=\csname maya@ff\the\maya@F\endcsname
            \maya@n=\maya@GlyphNum\multiply\maya@n by 5
            \immediate\write\maya@o{userdict begin MayaDict begin
                \the\maya@n\space \the\maya@F\space AddGlyphs}%
            \maya@io
            \immediate\write\maya@o{end end end end}%
            \ifnum\maya@GlyphNum>0 \maya@Message%
                {mayaps::WARNING: The following glyphs are not found in
                   `\csname maya@ff\the\maya@F\endcsname'}%
                \message{\space}
                {\loop
                \ifnum\maya@GlyphNum>0
                    \global\advance\maya@GlyphNum-1
                    \message{\csname maya@\the\maya@F
                        @\the\maya@GlyphNum\endcsname}
                \repeat}
                \maya@Message{}
            \fi
        \fi
        \expandafter\xdef\csname maya@\the\maya@F @GlyphNum\endcsname{0}%
        \advance\maya@F by 1
    \ifnum\maya@F<\maya@FNum
    \repeat
  \fi
}
%
%
%   To prepend maya@output to output, we adapt here the macro
%   \Prepend#1(to:)#2 from [V.Eijkhout, TeX by topics, Sect. 14.5.1]
%
\edef\Maya@act{\noexpand\output={\the\maya@output \the\output}}\Maya@act
%
%===================================================================
%
% This is a macro to add/replace a new glyph from an eps file.
% It is called by
%
% \mayaAddGlyph[(L)(U)(R)(D)(S)]{GlyphName}AC{FileName}
%
% where:
%  [(L)(U)(R)(D)(S)] (optional) is an affix modification vector. It is composed
%                               of four modification sequences
%                               (L left, U up, R right, D down, S single),
%                               each sequence being composed of the characters
%                               |, ', -, +, *, ?, R, and r whose meanings are:
%
%                               | '  symmetry with respect to the vertical axis
%                               -    symmetry with respect to the horizontal axis
%                               +    180 degree rotation (composition of | and -)
%                               R ?  rotation by 90 degree counterclockwise
%                               r *  rotation by 90 degree clockwise
%
%               Default value:  [()(|r)(|)(R)()]
%
%    GlyphName      (required)  The glyph name as it will appear in the argument
%                               of the macro \maya{...}. Any sequence of letters
%                               A..Z,  a..z  and digits  0..9
%
%    AC             (required)  A (for an affix) or C (for a central element)
%
%    FileName       (optional)  the name of the eps file
%
%               Default value:  GlyphName.eps
%
% Examples:
%  \mayaAddGlyph{foo}C                 %Add central element  foo  from  foo.eps
%
%  \mayaAddGlyph{aa1}A{1.eps}          %Add affix  aa1  from  1.eps
%
%  \mayaAddGlyph[()(r)(|)(R)()]{aa2}A  %Add affix  aa2  from  aa2.eps  with
%                                      %non-standard orientation when it is
%                                      %attached on the top of a central element
%
\def\mayaAddGlyph#1#2 {\ifmaya@AG
    %  check if epsf.tex is loaded. To do it, we check
    %  if the macro \epsfgetbb is defined (see TeXbook Ex7.7)
    \expandafter\ifx\csname epsfgetbb\endcsname\relax
        \input epsf
        \expandafter\ifx\csname epsfgetbb\endcsname\relax
            \maya@Message{mayaps: cannot find `epsf.tex'.
                \noexpand\mayaAddGlyph\space cannot work without}%
            \maya@Message{ \space\space\space\space\space\space\space this file.
                All \noexpand\mayaAddGlyph\space commands will be ignored.}%
            \maya@Message{}\maya@AGfalse\fi\fi
    \expandafter\ifx\csname maya@ff\the\maya@F\endcsname\relax\maya@AGfalse\fi
    \ifmaya@AG{%
        \def\lbracket{[}\def\a{#1}%
        \ifx\a\lbracket\maya@AG[#2 @\else\maya@AG[]{#1}#2 @\fi}\fi\fi}
\def\maya@AGwarn#1{\maya@Message{mayaps::mayaAddGlyph::WARNING: #1.
    Ignore it.}\maya@Message{}}%
\begingroup\catcode`\{=12 \catcode`\}=12 \catcode`\<=1 \catcode`\>=2
    \xdef\maya@lbrace<{>\xdef\maya@rbrace<}>\endgroup
\def\maya@write#1 @{\immediate\write\maya@o{#1}}%
\def\maya@AG[#1]#2#3#4 @{%\message{[#1][#2][#3][#4]}
    \def\e{}\def\modif{#1}\def\g{#2}\def\ac{#3}\def\fn{#4}%
    \ifx\e\fn\def\fn{#2.eps}\fi
    \openin\maya@i=\fn\relax
    \ifeof\maya@i\maya@AGwarn{cannot open `\fn'}
    \else
        \ifx\e\modif\else\maya@CheckBP{#1}\ifmaya@\else
            \def\modif{}\maya@AGwarn%
                {#2: bad parentheses inside [...]}
        \fi\fi
        \def\a{a}\def\A{A}\def\c{c}\def\C{C}%
        \ifx\ac\a\maya@true\else\ifx\ac\A\maya@true\else\ifx\ac\c\maya@false
            \else\ifx\ac\C\maya@false\else\maya@AGwarn{bad A/C key;
                 treated as `C'}\maya@false\fi\fi\fi\fi
        \epsfgetbb{\fn}
        \maya@n=\epsfurx\advance\maya@n-\epsfllx
        \maya@d=\epsfury\advance\maya@d-\epsflly
        \immediate\write\maya@o{userdict begin MayaDict begin 5 \the\maya@F
            \space AddGlyphs/a\g\space\ifmaya@ true\else false\fi\space def/w\g
            \space\the\maya@n\space def/h\g\space\the\maya@d\space def}%
        \ifx\e\modif\else\immediate\write\maya@o{/A\g[\modif]def}\fi
        \immediate\write\maya@o{/m\g\maya@lbrace save}%
        \openin\maya@i=\fn\relax
        %            a simplified version of the code from \maya@io:
        {%
            \def\e{\par}\maya@iOKtrue\chardef\other=12
            \def\do##1{\catcode`##1=\other}\dospecials \catcode`\ =10
            \loop
                \read\maya@i to \maya@line 
                \ifeof\maya@i \maya@iOKfalse
                \else
                    \ifx\e\maya@line \else
                        \expandafter\maya@AGaux\maya@line . \\%
                        \ifmaya@c\else
                            \expandafter\maya@write\maya@line @\fi\fi\fi
            \ifmaya@iOK
            \repeat
        }%
        \closein\maya@i
        \immediate\write\maya@o{restore\maya@rbrace bind def end end end end}%
        \maya@gn={\g}%
        \expandafter\xdef\csname maya@\the\maya@F @@\the\maya@gn\endcsname{}%
        \mayaGlobalDefine{#2}{#2}\fi}%
\long\def\maya@AGaux#1#2 \\{%
    \ifx#1\maya@P\maya@ctrue\else\maya@cfalse\fi}%
%===========================================================================
% Import a glyph from another font. Calling sequence:
%
%   \mayaImport[(L)(U)(R)(D)(S)]{NewName}\font{OldName}AC
%
% where:
% [(L)(U)(R)(D)(S)] (optional) same as in \mayaAddGlyph
%                              by default the value is inherited
%                              from the imported glyph
%
% NewName           (required)
%
% \font             (required) the font to import from
%                              as appear in \mayaFont\font=...
%
% OldName           (required) the glyph name in \font
%
% AC                (optional) A or C (affix/central)
%
\def\mayaImport#1#2 {{\let\E=\expandafter\def\e{}\maya@AGtrue%
  \let\l=\maya@lbrace\let\s=\space\def\lbracket{[}\def\a{#1}%
  \ifx\a\lbracket\maya@Imp[#2 @\else\maya@Imp[]{#1}#2 @\fi}}
\def\maya@Imp[#1]#2#3#4#5 @{\def\modif{#1}\def\ac{#5}%
  \def\u{#4\s\the\maya@n\s imp}\def\v{\maya@rbrace def}\def\t{\u1\v}%
  \E\xdef\csname maya@\the\maya@F @@#2\endcsname{}\maya@CheckBP{#1}%
  \ifmaya@\else\def\modif{}\maya@Message
    {mayaps::mayaImport::WARNING: #2: bad parentheses in `[#1]'}\fi
  {#3\maya@GetGlyphNames{#4}\global\maya@n=\maya@F}\def\a{a}\def\A{A}%
  \ifnum\maya@n=\maya@F\def\o{#2}\def\n{#4}\ifx\o\n\maya@AGfalse\fi\fi
  \ifmaya@AG\mayaGlobalDefine{#4}{#4}%
    \immediate\write\maya@o{userdict begin MayaDict begin 5 \the\maya@F\s
      AddGlyphs/A#2\ifx\modif\e\l mdf0/A\t\else[\modif]def\fi}%
    \immediate\write\maya@o{/w#2\l1/w\t/h#2\l1/h\t/m#2\l(#4)/m\u\v/a#2%
      \ifx\ac\e\l false/a\t
      \else\s\ifx\a\ac true\else\ifx\A\ac true\else false\fi\fi\s def\fi}%
    \immediate\write\maya@o{end end end end}
  \else\maya@Message{mayaps::mayaImport::WARNING: self-import #2->#4}\fi}%
%=================================================================----------
%
\openin\maya@i=red89.tex
\ifeof\maya@i
  \maya@Message{mayaps::WARNING: Can not find red89. The macros}%
  \maya@Message{\space\space\space\space\space\space\space\space\space
     \noexpand\mayaRed,\space\noexpand\mayaBW,\space\noexpand\codexBW
     \space are not available.}%
\else
  \closein\maya@i
  \input red89
\fi
%
\ifx\mayaNoPreloadedFont\mayaUndefinedMacro\relax
  \openin\maya@i=codex.mpf
  \ifeof\maya@i
    \maya@Message{mayaps::WARNING: Can not open codex.mpf. No Maya font is preloaded.}%
  \else
    \closein\maya@i
    \mayaFont\codex=codex
    \codex
    \openin\maya@i=red89.tex
    \ifeof\maya@i\relax\else\closein\maya@i\mayaRed\fi
    \mayaSize{12mm}                 % Default size recommended by Bruno Delprat
  \fi
\fi
%
\catcode`\@=12
%
\mayaRGB{0.85 0.10 0.04}r       % The color of red glyphs
%
\endinput
